\documentstyle[11pt]{article}
\setlength{\topmargin}{-.8in}
\addtolength{\textheight}{1.8in}
\addtolength{\textwidth}{\evensidemargin}
\addtolength{\textwidth}{\oddsidemargin}
\setlength{\oddsidemargin}{.25in}
\setlength{\evensidemargin}{.25in}
\addtolength{\textwidth}{-1.0\oddsidemargin}
\addtolength{\textwidth}{-1.0\evensidemargin}
\begin{document}
\setlength{\baselineskip}{14pt}

\begin{center}
{\Large  Call for Participation}\\ 
{\LARGE \bf The First DIMACS International \\
Algorithm Implementation Challenge: \\
Network Flows and Matching }
\end{center} 
\vspace{.25in}

The Center for Discrete Mathematics and Theoretical Computer Science
(DIMACS) invites participation in an international Implementation Challenge 
to find and evaluate efficient and robust implementations of algorithms
for the following problems: {\bf Minimum-cost Flows}, {\bf Maximum Flows}, 
{\bf Assignment}, and {\bf Nonbipartite Matching}. 

The Implementation Challenge will take place between November 1990 and
August 1991.  Participants are invited to carry out research 
projects  related to these problem areas and to present research papers 
at a DIMACS workshop to be held in Fall of 1991.  Best paper awards
will be presented in several categories.  Workshop proceedings will 
be published. With author's permission, the most successful implementations
will be collected for distribution on floppy disk. 

\paragraph{Advisory Board.}  
A committee of DIMACS members will set policy guidelines and 
provide general direction for the Implementation Challenge. 
 Committee members are:  
\vspace{.1in} 

\begin{tabular}{ll}
Mike Grigoriadis & Rutgers University \\
David Johnson & AT\&T Bell Laboratories \\
Cathy McGeoch & DIMACS Visiting Fellow/Amherst College (Chairperson)\\
Clyde Monma & Bell Communications Research \\
Bob Tarjan & Princeton University \\
\end{tabular}

\paragraph{Research Projects.} 
Although many new algorithms for flow and matching problems have appeared 
recently in theoretical papers, very little is known about how theoretical
analyses compare to real performance:  our goal is to provide a 
catalyst for experimental research in these problem areas.  
Participants may wish to implement algorithms for evaluation, to 
build interesting input generators, or to develop implementations
tuned for newer architectures.  Projects may involve either public domain
or proprietary codes. 

\paragraph{DIMACS Support.} 
DIMACS will provide benchmark instances for each problem and
support tools and guidelines for research projects.  DIMACS 
facilities will provide a clearing-house for exchange of 
programs and input generators and for communication among researchers.  
DIMACS can provide neither financial support for research projects nor 
machine cycles for the experiments.  

\paragraph{How to Participate} 

For more information about participating in the Implementation Challenge, 
send a request for the documents  {\em General Information} 
and {\em Problem Definitions and Input Specifications} to 
{\bf netflow@dimacs.rutgers.edu}.  Request either 
\LaTeX$\;$ format (sent through email) or hard copy 
(sent through U. S. Mail), and include your return address as 
appropriate.  One goal of the Challenge is to evaluate the 
suitability of the Internet
for cooperative projects of this kind. {\em All correspondence regarding
the implementation challenge will take place via Internet.} 

\end{document}
